\chapter[Summary and Future Work][Summary and Future Work]{Summary and Future Work} \label{ch:Summary}

\section{Main Contributions}
%
\begin{enumerate}
\item \textit{NFRHT Between Layered Spheres:} I developed the first numerically exact model for NFRHT between two coated spheres and used it to investigate the impact of coatings on the spectrum of NFRHT. I observed a number of interference-based peaks in the spectrum of NFRHT between two metallic spheres with polar material coatings. The peaks were reminiscent of my earliest work involving NFRHT between two homogeneous silicon carbide spheres.\cite{Czapla2014} These similarities were due to silicon carbide's low absorption allowing waves to enter into each sphere, reflect off the backside and constructively/destructively interfere with other waves, in a similar fashion to waves entering a thin coating and reflecting off the metallic core. I also showed that layered spheres composed of different materials which support SPhPs can result in enhanced NFRHT, above that of either material alone. (See Refs. \citenum{Czapla2017}, \citenum{Czapla2014}, and \citenum{Czapla2017b}.)
\item \textit{NFRHT in Chains of Spheres:} I extended my two sphere work to include any number of coated spheres in a linear chain. I can simulate both sphere-sphere and sphere-environment radiative transfer. Using this work, I was able to issue a warning regarding the distance-dependence of sphere-environment heat transfer to any experimentalist who wishes to conduct sphere-sphere NFRHT experiments using a microcantilever. I also proposed a solution to mitigate the potential risk. Furthermore, I moved software development to GitHub\cite{Narayanaswamy2018} to lower the barrier of entry for other researchers. (See Ref. \citenum{Czapla2018}.)
\item \textit{New Optical Properties:} I published the first broadband optical data for PDMS in the mid-infrared portion of the electromagnetic spectrum. I used those properties to explore the potential of PDMS for radiative cooling applications, another situation that requires tailoring the spectrum of radiative transfer, and demonstrated numerically that PDMS thin films atop aluminum substrates could theoretically reach equilibrium temperatures of 12\si{\celsius} below ambient when exposed to the night sky. (See Refs. \citenum{Srinivasan2016} and \citenum{Czapla2017a}.)
\end{enumerate}


\section{Future Work}
%
\begin{enumerate}
\item \textit{Sphere-Plane Heat Transfer:} Otey et al.\cite{Otey2011} computed the NFRHT between a homogeneous sphere and substrate using a method analogous to the interior method. In doing so, they translated back and forth between vector spherical and cylindrical (required for the plane) waves. It would be interesting to see if the same result could be obtained using an exterior method approach and asymptotic expressions for the spherical Bessel and Hankel functions in the limit as one sphere gets very large. Previous work in the Swamy Group by Sasihithlu and Narayanaswamy\cite{Sasihithlu2014} considered only homogeneous spheres with large size disparity ($\rho_{2} \lesssim 40 \rho_{1}$), not a true $\rho_{2}/\rho_{1} \rightarrow \infty$ case, and requires a convergence analysis to prove that the sphere-sphere NFRHT reaches a steady value which can approximate sphere-plane NFRHT. By taking the approach of Ref. \citenum{Sasihithlu2014}, we should get accurate sphere-environment heat transfer but perhaps not an accurate plane-environment heat transfer. My proposed method should bypass those shortcomings and give access to NFRHT between coated objects in the process.
\item \textit{Higher-Order Discrete Dipole Approximations:} The method outlined in this work should reproduce the discrete dipole approximation (DDA) for two spheres when their radii are small and $l_{\mathrm{max}} = \nu_{\mathrm{max}} = 1.$ Firstly, this should be verified. Secondly, the linear system of scattered field coefficients should be solved explicitly in the case of $l_{\mathrm{max}} = \nu_{\mathrm{max}} =2, 3, 4,... $ until it is no longer feasible. This would provide higher order corrections which could yield DDA models applicable to larger spheres.
\item \textit{Scattered Field Coefficient Bottleneck:} The greatest bottleneck in calculating NFRHT between closely-spaced wavelength-sized spheres is solving for the scattered field coefficients using their linear system. The number of terms for convergence is too great when the spheres are large, very close, or very far. Examining the results of the higher-order DDA could shed insight into a solution to the linear system which doesn't require matrix inversion. Richardson extrapolation also has the potential to accelerate the convergence of the infinite series.
%\item \textit{NFRHT Python Library:} The NFRHT community would benefit greatly if it could coallesce around a common set of open-source numerical tools. Could include future innovations such as... 
\item \textit{Experimental Validation:} As of yet, no two sphere NFRHT experiments have taken place, let alone multiple-body experiments. Experimental NFRHT has lagged behind theoretical NFRHT, and that mismatch should be corrected, both to validate my models and to innovate on existing experimental techniques so we can move toward producing devices which exploit NFRHT phenomena.
\end{enumerate}


%Coding - service to the community. Python library that can translate between all systems. Precompiled Fortran or C code in a python library. move beyond chain. Include common approximations and dielectric functions. Basically, become a software engineer.

%Cylinder-cylinder exterior method. Mostly for completeness.

%Using chain of progressively smaller spheres as a proxy for a sharp tip over a surface (large sphere)

