% This is the abstract of my dissertation.

\thispagestyle{empty} % No page number in entire abstract
\begin{center}
ABSTRACT

Near-Field Radiative Heat Transfer in Linear Chains of Multilayered Spheres

Braden Czapla
\end{center}

Thermal radiation is ubiquitous to all matter at finite temperature and controlling the radiative nature of that matter has been a key enabling factor in the development of several recent technologies, such as thermal diodes, thermal antennae, thermophotovoltaics, heat-assisted magnetic recording, and contactless cooling in microelectromechanical systems. At the micro/nano-scale, thermal radiation does not reliably behave in the way Planck's blackbody law predicts, due to near-field effects such as the diffraction, interference, and tunneling of light. In fact, the so-called blackbody limit can routinely be broken by several orders of magnitude when objects of dimensions or separation distances much smaller than the peak thermal wavelength (approximately 10 \si{\micro\meter} at room temperature) exchange thermal radiation. A deeper theory is required to understand near-field thermal radiation: Maxwell's equations. Maxwell's equations allow for a direct connection between the thermally induced current fluctuations and radiative transfer. 

In this dissertation, I investigate radiative transfer among spherical bodies aligned in a linear chain. The chain may be composed of any number of spheres, and the spheres themselves may be composed of any linear isotropic material, may be of any size and separation distance, and may each have any number of spherically symmetric layers. Using a dyadic Green's function formalism, I derive numerically exact formulas for heat transfer between pairs of spheres in the chain and between any sphere in the chain and its environment.

My work clearly demonstrates that adding coatings to spherical objects can drastically impact the spectrum of radiative transfer, enhancing or diminishing it in various cases. This degree of tailoring makes coated spheres a flexible, yet unexplored, platform for future experiments in near-field radiative heat transfer. My work also demonstrates that, in an experiment measuring the distance dependent heat transfer between two spheres, heat transfer from the spheres to their environment can also have a strong distance dependence, which must be considered carefully when designing an experiment and analyzing its results. This demonstrates a cautious but optimistic outlook for the near-field radiative heat transfer community moving beyond traditional plane-plane and sphere-plane experimental configurations. 